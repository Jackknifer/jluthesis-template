% =========================================================================
% 吉林大学本科毕业论文(设计)LaTeX 模板
% 文件名:main.tex
% =========================================================================

\documentclass{jluthesis}

\begin{document}

	% =====================================================================
	% 1. 封面与承诺书 (Cover & Commitment)
	% =====================================================================
	% 生成封面 (配置信息在 jluthesis.cfg 中修改)
    \makecover
	% 插入承诺书 (请确保 docs/commitment.pdf 文件存在)
	\includepdf{docs/commitment.pdf}
	
	% =====================================================================
	% 2. 前置部分 (Front Matter)
	% =====================================================================
	% 页码从摘要开始使用罗马数字 (I, II, ...)
	\pagenumbering{Roman}

	% ---------------------------------------------------------------------
	% 中文摘要 (Chinese Abstract)
	% ---------------------------------------------------------------------
    \begin{cabstract}
        这里是中文摘要部分。请在这里简要介绍你的论文内容,例如:本文主要研究了......
        
        % 关键词:词与词之间用分号分隔
        \ckeywords{关键词1;关键词2;关键词3}
    \end{cabstract}
	
	% ---------------------------------------------------------------------
	% 英文摘要 (English Abstract)
	% ---------------------------------------------------------------------
    \begin{eabstract}
        Here is the abstract in English. This paper focuses on...
        
        % Keywords: Separated by semicolons
        \ekeywords{Keyword1; Keyword2; Keyword3}
    \end{eabstract}
	
	% ---------------------------------------------------------------------
	% 目录 (Table of Contents)
	% ---------------------------------------------------------------------
	% 自动生成目录
    \makeTOC
	
	% =====================================================================
	% 3. 正文部分 (Main Body)
	% =====================================================================
	% 初始化正文格式 (重置页码为阿拉伯数字 1, 2, ...,设置页眉)
    \startmainmatter
	
	% 第一章 (Chapter 1)
	\section{引言}
	
	\subsection{研究背景与意义}
	在此处撰写研究背景。可以使用引用 \cite{ref1}。
	
	\subsection{国内外研究现状}
	在此处撰写文献综述。例如,文献 \cite{ref2} 指出......
	
	\subsection{本文主要工作}
	本文的主要贡献如下:
	\begin{enumerate}[(1)]
		\item 贡献点一......
		\item 贡献点二......
	\end{enumerate}
	
	% 第二章 (Chapter 2)
	\newpage
	\section{预备知识}
	
	\subsection{相关理论基础}
	介绍论文用到的数学模型。
	
	% 公式示例 (Equation Example)
	\begin{equation}
		f(x) = \frac{1}{\sqrt{2\pi}\sigma} e^{-\frac{(x-\mu)^2}{2\sigma^2}}
		\label{eq:normal_dist}
	\end{equation}
	
	% 第三章 (Chapter 3)
	\newpage
	\section{模型建立与分析}
	
	\subsection{算法设计}
	详细描述你的算法。

	% 代码插入示例 (Code Listing Example)
	% 更多语言支持可参考 listings 宏包文档
	\begin{lstlisting}[language=Matlab, caption={Matlab 代码示例}, captionpos=b]
% 这里是代码示例
function y = my_function(x)
	y = x^2 + 2*x + 1;
end
	\end{lstlisting}
	

	% Python 代码插入示例
	\begin{lstlisting}[language=Python, style=general, caption={Python 代码示例}]
def hello_world():
    # 这是一个 Python 函数示例
    print("Hello, LaTeX!")
    return True
	\end{lstlisting}

	% C++ 代码插入示例
	\begin{lstlisting}[language=C++, style=general, caption={C++ 代码示例}]
#include <iostream>
using namespace std;

int main() {
    # 这是一个 C++ 主函数
    cout << "Hello World!";
    return 0;
}
	\end{lstlisting}
	
	% 第四章 (Chapter 4)
	\newpage
	\section{数值模拟与分析}
	
	\subsection{实验设置}
	实验说明。
	
	% 图片插图示例 (Figure Example)
	% 图片文件请放在 images/ 目录下
    \begin{figure}[H]
        \centering
        \includegraphics[height=6cm,width=8.5cm]{example.jpg} 
        \caption{一个示意图}
        \label{figure:example}
    \end{figure}
	
	\subsection{结果分析}
	如表 \ref{tab:example} 和图 \ref{figure:example} 所示......

	% 表格插入示例 (Table Example)
	% 推荐使用三线表 (booktabs 宏包)
	\begin{table}[H]
		\centering
		\caption{实验数据对比表}
		\label{tab:example}
		\begin{tabular}{ccc}
			\toprule
			指标 & 算法A & 算法B \\
			\midrule
			准确率 & 95.2\% & 96.8\% \\
			召回率 & 94.5\% & 95.1\% \\
			F1分数 & 94.8\% & 95.9\% \\
			\bottomrule
		\end{tabular}
	\end{table}
	
	% 第五章 (Chapter 5)
	\newpage
	\section{总结与展望}
	
	本文总结了......
	
	% =====================================================================
	% 4. 结尾部分 (Back Matter)
	% =====================================================================
	
	% 参考文献 (References)
	% 自动引用 references.bib 中的内容
    \makebibliography
	
	% 致谢 (Acknowledgment)
    \begin{acknowledgment}
	    感谢导师和同学的帮助......
    \end{acknowledgment}
	
\end{document}
